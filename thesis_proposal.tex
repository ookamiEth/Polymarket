\documentclass[11pt,a4paper]{article}
\usepackage[utf8]{inputenc}
\usepackage[margin=1in]{geometry}
\usepackage{amsmath}
\usepackage{amssymb}
\usepackage{url}
\usepackage{hyperref}
\usepackage{setspace}

% Spacing settings to match baseline style
\setstretch{1.3}
\setlength{\parskip}{6pt plus 2pt minus 1pt}

\begin{document}

\vspace*{0.5cm}

\begin{center}
\Large \textsc{Bachelor Thesis at the Department of\\
Information Technology and Electrical\\
Engineering}

\vspace{0.6cm}

\Large \textsc{Fall Semester 2025}

\vspace{1.8cm}

\LARGE Leo Gierhake

\vspace{2cm}

\Large \textbf{Machine Learning-Based Binary Options Pricing\\
Model for Prediction Markets}

\vspace{2cm}

\large September 1, 2025

\end{center}

\vspace{1.2cm}

\noindent
\begin{tabular}{@{}ll@{}}
Advisors: & [TBD] \\
Supervisor: & [TBD] \\
\\
\end{tabular}

\vspace{0.3cm}

\noindent
\begin{tabular}{@{}ll@{}}
Handout: & [TBD] \\
Due: & [TBD] \\
\end{tabular}

\vspace{0.8cm}

\noindent The final report will be submitted in electronic format. All copies remain property of the Distributed Computing Group.

\newpage

\section{Introduction}

Prediction markets have emerged as powerful information aggregation mechanisms, allowing participants to trade contracts whose payoffs depend on future events. Polymarket, one of the largest decentralized prediction market platforms, offers ultra-short-dated binary options on cryptocurrency and equity price movements with durations as brief as 15 minutes. These instruments function as continuous probability estimates: a contract paying \$1 if the underlying closes higher than its opening price should theoretically trade at the market's consensus probability of that outcome. However, these instruments lack the rigorous pricing frameworks that exist for traditional derivatives, and the applicability of established models like Black-Scholes to 15-minute binary contracts remains unexplored.

The central question of this thesis is: can we develop a theoretically rigorous and practically applicable pricing model for ultra-short-dated binary options in prediction markets? While classical derivatives pricing theory provides a foundation, the extreme time scales involved pose unique challenges. To inform model development, this work leverages implied volatility data from liquid cryptocurrency derivatives exchanges such as Deribit, which capture sophisticated market participants' expectations of future price uncertainty. However, fundamental differences in market microstructure and potential temporal mismatches between longer-dated options markets and ultra-short binaries must be addressed.

This thesis proposes a systematic approach to binary option pricing for prediction markets. Beginning with a Black-Scholes framework adapted for binary payoffs, the work will investigate the integration of multi-scale realized volatility to address potential temporal signal staleness. Machine learning techniques will be explored to capture non-linear relationships and market microstructure effects that classical models cannot represent. Beyond demonstrating technical feasibility, this work aims to quantify fundamental limits: How much information can traditional options markets provide to prediction market pricing? What role does machine learning play in bridging the gap between theoretical pricing and market reality? The answers have implications for market efficiency, information aggregation, and the design of next-generation forecasting systems.

\section{Project Description}

\subsection{Task description}

This thesis involves ten main tasks to build and test a pricing model for prediction markets:

\begin{itemize}

\item \textbf{Task I: Data Infrastructure} Build and deploy data collection systems for three sources: historical trade data from closed prediction market contracts, real-time orderbook streaming services capturing live market data, and options market data from cryptocurrency derivatives exchanges. Implement efficient storage and querying systems to support large-scale empirical analysis.

\item \textbf{Task II: Data Processing} Develop processing pipelines to handle large datasets using memory-optimized techniques suitable for resource-constrained environments. Construct cleaned datasets including options quotes with computed implied volatilities, high-frequency spot price data resampled at one-second intervals, and contract schedules mapping prediction market trading periods.

\item \textbf{Task III: Feature Engineering} Design and implement a comprehensive feature space for model training. Features will include multi-scale realized volatility measures computed at different time horizons, market microstructure signals capturing momentum, price jumps, reversals, and statistical properties of price dynamics, and contextual variables such as time remaining until contract expiry and data staleness indicators.

\item \textbf{Task IV: Baseline Model} Implement a Black-Scholes pricing framework adapted for binary options, using implied volatilities from options markets as volatility estimates. Evaluate baseline model performance on historical data and conduct diagnostic analysis to identify primary sources of pricing error and model limitations at ultra-short time scales.

\item \textbf{Task V: Volatility Integration} Investigate approaches to address temporal mismatches between options market implied volatilities and ultra-short-dated binary pricing. Develop blending methodologies that combine implied volatility with recent realized volatility computed from high-frequency price data, with adaptive weighting based on data freshness and market conditions.

\item \textbf{Task VI: Machine Learning Enhancement} Train statistical and machine learning models to capture systematic pricing errors not addressed by classical theory. Implement Ridge regression for interpretable linear corrections, followed by gradient-boosted tree models to capture non-linear relationships and regime-dependent effects. Evaluate incremental improvements from each modeling approach.

\item \textbf{Task VII: Validation and Analysis} Conduct comprehensive model validation including calibration analysis to assess probability forecast quality, performance evaluation across different market regimes, and feature importance analysis to identify key drivers of pricing accuracy. Document results with clear visualizations and statistical tests.

\item \textbf{Task VIII: Event Volatility Decomposition} Address the challenge that near-term options (expiring 1--2 days ahead) incorporate upcoming macro event risk rather than base structural volatility. Integrate a traditional macro event schedule covering Fed meetings, CPI releases, employment reports, and GDP announcements. Develop methodology to decompose implied volatility into base volatility and event premium components, extracting the structural volatility signal by filtering event-driven spikes. Integrate base volatility estimates into the pricing model to improve accuracy.

\item \textbf{Task IX: Multi-Asset Expansion} Extend the complete modeling framework from Bitcoin to Ethereum, Solana, and XRP, all of which have 15-minute binary contracts on Polymarket. Adapt data collection pipelines for multi-asset spot prices and options data, retrain and calibrate models for each asset's unique characteristics, and conduct cross-asset validation. Identify asset-specific considerations versus universal pricing patterns that generalize across cryptocurrencies.

\item \textbf{Task X: Live Market Making System} Deploy the pricing model into a live execution system for real-world validation. Build trading infrastructure connecting model probability outputs to Polymarket order placement, implement market making strategies that provide liquidity at model-fair prices, and incorporate basic risk management and position limits. Monitor and document all trading activity including executed trades, profit and loss, and model performance under live market conditions.

\end{itemize}

\section{Milestones}

\begin{enumerate}

\item \textbf{M1 (Tasks I \& II):} Data infrastructure operational and validated. All three data collection pipelines deployed: historical trade data collection, real-time orderbook streaming services, and options market data acquisition. Processed datasets ready for empirical analysis.

\item \textbf{M2 (Task III):} Feature engineering framework implemented and tested. Multi-scale volatility measures, microstructure signals, and contextual features computed at high frequency and validated for quality and correctness on sample contracts.

\item \textbf{M3 (Task IV):} Baseline pricing model implemented and benchmarked. Black-Scholes binary option framework evaluated on historical data. Diagnostic analysis performed to identify primary error sources and model limitations.

\item \textbf{M4 (Task V):} Volatility integration approach implemented and evaluated. Blending methodologies combining implied and realized volatility tested across different market conditions. Improvements over baseline quantified and documented.

\item \textbf{M5 (Task VI):} Machine learning models trained and validated. Ridge regression and gradient-boosted tree models implemented. Incremental performance gains evaluated and compared against baseline and intermediate approaches.

\item \textbf{M6 (Task VII):} Comprehensive validation of core pricing models completed. Calibration analysis, regime-dependent performance evaluation, and feature importance studies documented for baseline and enhanced models.

\item \textbf{M7 (Task VIII):} Event volatility decomposition framework implemented and validated. Macro event schedule integrated, base volatility extraction methodology developed and tested on historical data. Improvements over raw implied volatility quantified and documented.

\item \textbf{M8 (Task IX):} Multi-asset modeling framework operational. Data pipelines extended to Ethereum, Solana, and XRP. Models trained and benchmarked for all assets. Cross-asset performance comparison and asset-specific findings documented.

\item \textbf{M9 (Task X):} Live market making system deployed and evaluated, thesis completed. Trading infrastructure operational with model-driven order placement on Polymarket. Live trading results documented including execution statistics, profit and loss, and real-time model performance metrics. Final thesis report written, presentation delivered, and reproducible code provided.

\end{enumerate}

\end{document}
